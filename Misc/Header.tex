\PassOptionsToPackage{english, main=english}{babel}

\documentclass[
	% -- opções da classe memoir --
	12pt,				% tamanho da fonte
	%openright,			% capítulos começam em pág ímpar (insere página vazia caso preciso)
	oneside,			% para impressão sem recto e verso. Oposto a twoside
	%twoside,			% para impressão em recto e verso. Oposto a oneside
	a4paper,			% tamanho do papel. 
	% -- opções da classe abntex2 --
	%chapter=TITLE,		% títulos de capítulos convertidos em letras maiúsculas
	%section=TITLE,		% títulos de seções convertidos em letras maiúsculas
	%subsection=TITLE,	% títulos de subseções convertidos em letras maiúsculas
	%subsubsection=TITLE,% títulos de subsubseções convertidos em letras maiúsculas
	sumario = tradicional,
	% -- opções do pacote babel --
	%french,			% idioma adicional para hifenização
	spanish,			% idioma adicional para hifenização
	brazil,				% idioma adicional para hifenização 
	english 			% o último idioma é o principal do documento
	]{abntex2}

\usepackage{titlesec}

\titleformat{\section}{\normalfont\fontsize{16}{16}\selectfont\bfseries}{\thesection}{1em}{}
\titleformat{\subsection}{\normalfont\fontsize{14}{16}\selectfont\bfseries}{\thesubsection}{1em}{}
\titleformat{\subsubsection}{\normalfont\fontsize{12}{16}\selectfont\bfseries}{\thesubsubsection}{1em}{}

\renewcommand{\arraystretch}{1.8}
% ---
% Pacotes básicos 
% ---
\usepackage{Misc/pucpr}	
\usepackage{helvet}			% Usa a fonte Latin Modern
%\usepackage{lmodern}			% Usa a fonte Latin Modern			
\usepackage[T1]{fontenc}		% Selecao de codigos de fonte.
\usepackage[utf8]{inputenc}		% Codificacao do documento (conversão automática dos acentos)
\usepackage{lastpage}			% Usado pela Ficha catalográfica
\usepackage{indentfirst}		% Indenta o primeiro parágrafo de cada seção.
\usepackage{color}				% Controle das cores
\usepackage{graphicx}			% Inclusão de gráficos
\usepackage{microtype} 			% para melhorias de justificação
% ---
\usepackage{subfig}             % Para subfiguras
\usepackage{url}                % Para sites da internet
\usepackage{multirow}
\usepackage{amsmath}
\usepackage{bm}
\usepackage{algorithm}
\usepackage[noend]{algpseudocode}
\usepackage{siunitx}
\usepackage{afterpage}
\usepackage{lscape}
\usepackage{amssymb}
\usepackage{yfonts}
\usepackage{enumitem}
\setlist{leftmargin=2cm}
\usepackage{pgfgantt}
\usepackage{longtable,pdflscape,booktabs}
\usepackage{lmodern}
\usepackage{algorithm}
\usepackage{multicol}
\usepackage{tikz}
\usetikzlibrary{calc}

\usepackage{pdfpages}

\usepackage{Misc/coffe4}

\xdefinecolor{mycolor}{RGB}{62,96,111} % Neutral Blue
\colorlet{bancolor}{mycolor}

\def\bancolor{mycolor}
\newenvironment{mybox}[3][]{%
    \begin{tikzpicture}[#1]%
        \def\myboxname{#3}%
        \node [draw,inner sep=1.5ex,text width=#2]% good options: minimum height, minimum width
            (BOXCONTENT) \bgroup\rule{0pt}{3ex}\ignorespaces
}{%
        \egroup;
        \node [right,inner xsep=1em,fill=bancolor!75,outer sep=0pt,text height=2ex,text depth=.5ex] (BOXNAME) 
            at ([shift={(-1em,0pt)}]BOXCONTENT.north west) {\myboxname};
        \fill[bancolor] (BOXNAME.north east) -- +(-1em,1em) -- +(-1em,0) -- cycle;
        \fill[bancolor] (BOXNAME.south west) -- +(1em,-1em) -- +(1em,0) -- cycle;
    \end{tikzpicture}
}


%% Watermark com data
% \usepackage{draft watermark}
% \usepackage{datetime}
% \ddmmyydate
% \SetWatermarkLightness{.9}
% \SetWatermarkText{Draft\string@\today}
% \SetWatermarkScale{.6}

% ---
% Pacotes de citações
% ---
\usepackage[english,hyperpageref]{backref}	 % Paginas com as citações na bibl
 \usepackage[alf,abnt-etal-list=0,abnt-etal-cite=2]{abntex2cite}%referências bibliográficas abnt

\renewcommand{\backrefpagesname}{Cited in page(s):~}
% Texto padrão antes do número das páginas
\renewcommand{\backref}{}
% Define os textos da citação
\renewcommand*{\backrefalt}[4]{
	\ifcase #1 %
		No citation in the text.%
	\or
		Cited in page #2.%
	\else
		Cited #1 times in pages #2.%
	\fi}%
\usepackage{acronym}

%=============================
% CUSTOMIZAÇÕES DOS CAPÍTULOS
%=============================

\makeatletter
\newcommand\thickhrulefill{\leavevmode \leaders \hrule height 1ex \hfill \kern \z@}
\setlength\midchapskip{10pt}
\makechapterstyle{VZ14}{
\renewcommand\chapternamenum{}
\renewcommand\printchaptername{}
\renewcommand\chapnamefont{\normalsize}
\renewcommand\printchapternum{%
\thickhrulefill\quad \chaptername
\space\thechapter\quad\thickhrulefill}
\renewcommand\printchapternonum{%
\par\thickhrulefill\par\vskip\midchapskip
\hrule\vskip\midchapskip
}
\renewcommand\chaptitlefont{\Huge\centering}
\renewcommand\afterchapternum{%
\par\nobreak\vskip\midchapskip\hrule\vskip\midchapskip}
\renewcommand\afterchaptertitle{%
\par\vskip\midchapskip\hrule\nobreak\vskip\afterchapskip}
}
\makeatother

\chapterstyle{VZ14}
% --- 
% CONFIGURAÇÕES DE PACOTES
% --- 

% ---



% informações do PDF
\makeatletter
\hypersetup{
     	%pagebackref=true,
		pdftitle={\@title}, 
		pdfauthor={\@author},
    	pdfsubject={\imprimirpreambulo},
	    pdfcreator={LaTeX with abnTeX2},
		pdfkeywords={abnt}{latex}{abntex}{abntex2}{trabalho acadêmico}, 
		colorlinks=true,       		% false: boxed links; true: colored links
    	linkcolor=black,          	% color of internal links
    	citecolor=black,        		% color of links to bibliography
    	filecolor=magenta,      		% color of file links
		urlcolor=blue,
		bookmarksdepth=4
}
\makeatother
% --- 

% --- 
% Espaçamentos entre linhas e parágrafos 
% --- 

% O tamanho do parágrafo é dado por:
\setlength{\parindent}{1.5cm}

% Controle do espaçamento entre um parágrafo e outro:
\setlength{\parskip}{0.2cm}  % tente também \onelineskip

% ---
% Compila o indice
% ---
%\makeindex
% ---

% Seleciona o idioma do documento (conforme pacotes do babel)
\selectlanguage{english}
%\selectlanguage{brazil}

% Retira espaço extra obsoleto entre as frases.
\frenchspacing 

%Customizações para Source

% \newcommand{\source}[1]{\caption*{Source: {#1}} }
%Aligned
\newcommand{\source}[1]{\vspace{-3pt} \caption*{ Source: {#1}} }