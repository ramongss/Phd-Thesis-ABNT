\chapter{Conclusions and Research Opportunities} \label{chap:conclusion}

% Summary of the objective of this thesis
The main objective of this thesis was to propose an effective hybrid forecasting framework that couples signal decomposition methods and \ac{STACK} approach to forecasting time series in real-world applications. The proposed methods were evaluated considering epidemiological and renewable energy time series. 
% describe the context of each application
In an epidemiological context, the \ac{COVID-19} cumulative cases time series had the urgency to forecast cases with efficiency and efficacy to better decision-making since the pandemic was a global public health concern. Also, the unknowing characteristics of the disease and the lack of information (related variables) at that time were challenging. For the renewable energy context, both wind energy generation and wind speed presented high fluctuations in their time series, primarily due to the intermittent behavior of the wind speed. This behavior led to the nonlinear and non-stationary time series pattern, making it difficult to forecast accurately. Hence, to deal with these challenges, this thesis aimed to develop efficient hybrid forecasting frameworks to predict time series as accurately as possible.
% brief outline of the thesis
In Chapter \ref{chap:introduction}, the contextualization, objectives, motivation, contributions, and research publications of this thesis are introduced. Next, in Chapter \ref{chap:theoretical}, the theoretical background of the concepts utilized in this work is detailed. Chapter \ref{chap:relatedworks} brings the recent related works regarding each application context. Further, Chapter \ref{chap:realworld} presented the dataset descriptions, methodologies, and contributions of each application, respectively. Last, in Chapter \ref{chap:results}, the results and discussions of the applications are debated.

% conclusions of application 1
Regarding the first application of this thesis, in Subsection \ref{sec:case_study1}, the study deals with the evaluation of the employment of signal decomposition approaches and the impact of the exogenous features in the learning process in the context of \textbf{COVID-19} cumulative cases forecasting. Considering the \textbf{RQ 1.1} - \textit{Can signal decomposition approaches enhance the performance of forecasting \ac{COVID-19} cumulative cases time series?}, it is possible to infer that \ac{CUBIST} coupled with the \ac{VMD} model are suitable tools to forecast \ac{COVID-19} cases for most of the adopted states once these approaches were able to learn the non-linearities inherent to the evaluated epidemiological time series. Also, \ac{BRNN} and \ac{SVR} models deserve attention for developing this task. Therefore, the ranking of models in all scenarios for Brazilian states is \ac{VMD}--\ac{CUBIST}, \ac{VMD}--\ac{BRNN}, \ac{SVR}, \ac{CUBIST}, \ac{VMD}--\ac{SVR}, \ac{BRNN}, \ac{VMD}--\ac{QRF}, \ac{QRF}, \ac{VMD}--\ac{KNN}, and \ac{KNN}, and for \ac{USA} states is \ac{VMD}--\ac{CUBIST}, \ac{BRNN}, \ac{CUBIST}, \ac{SVR}, \ac{VMD}--\ac{BRNN}, \ac{VMD}--\ac{SVR}, \ac{VMD}--\ac{QRF}, \ac{QRF}, \ac{KNN}, and \ac{VMD}--\ac{KNN}. Also, looking for \ac{COVID-19} forecasts six days ahead, hybrid models are more suitable tools than non-decomposed models. Further to answer \textbf{RQ 2} - \textit{How does the use of exogenous features impact performance improvement when forecasting \ac{COVID-19} cumulative cases time series?}, it was observed that climatic variables, such as temperature and precipitation, indeed influence increasing the accuracy when predicting \ac{COVID-19} cases, wherein some cases climate inputs reached up to 50\% of importance in the forecasting model. To answer \textbf{RQ 1.1} and \textbf{RQ 2} the specific objectives \textbf{\ref{obj_a}} and \textbf{\ref{obj_c}} were achieved.

% conclusions of application 2
Next, in Subsection \ref{sec:case_study2}, the second application concerning wind energy generation forecasting aimed to combine the signal decomposition methods with \ac{STACK} approach to investigate the forecasting performance improvement of this hybrid strategy. Further, evaluate preprocessing methods dealing with the time series data. According to the results, it can be concluded that regarding \textbf{RQ 1.2} - \textit{Can signal decomposition approaches enhance the performance of forecasting wind energy generation time series?} and \textbf{RQ 3.1} - \textit{What is the improvement achieved by employing the \ac{STACK} approach coupled with signal decomposition approaches over non-decomposed models when forecasting wind energy generation time series?}, signal decomposition models outperform decomposed, stacking-ensemble, and non-decomposed models. Due to the divide-and-conquer scheme, the stacking-ensemble learning enhanced the accuracy of the weak \ac{CEEMD} models by combining and using them to forecast with a strong model, giving more robustness and stability to the model, even though some weak models presented high errors and low accuracy. Regarding \textbf{RQ 4} - \textit{Can preprocessing methods applied to the time series improve the forecasting performance of the decomposition-ensemble learning strategy?}, the preprocessing techniques improved the performance of the forecasts. They transform the time series making the learning process of the models easier due to their characteristics of filtering highly correlated features or reducing the number of dimensions and then removing multicollinearity from the features. It can be highlighted that the \ac{BC} method outperformed other models in five out of nine scenarios, according to Table \ref{tab:DMtest}. By answering theses \ac{RQ}s, the specific objectives \textbf{\ref{obj_a}}, \textbf{\ref{obj_b}}, and \textbf{\ref{obj_d}} were achieved.

% conclusions of application 3
Last, after the satisfactory results of the forecasting framework proposed in the previous application (Subsection \ref{sec:case_study2}), the third application, presented in Subsection \ref{sec:case_study3}, willed to propose an upgrade in the hybrid forecasting framework by adding a multi-stage signal decomposition strategy. Hence, this new framework aimed to evaluate the performance of the multi-stage signal decomposition strategy and its combination with the \ac{STACK} approach in the wind speed forecasting context. The analysis of the results concludes that regarding \textbf{RQ 1.3} - \textit{Can signal decomposition approaches enhance the performance of forecasting wind speed time series?} and \textbf{RQ 5} - \textit{Can the multi-stage signal decomposition strategy outperform the single-stage signal decomposition approach?}, the dual decomposition approach outperformed the single and non-decomposition approaches due to the \ac{VMD} technique that extracted the trend component from the original time series and the \ac{SSA} approach denoised the remaining signal. That enables the training models to deal with the high and low frequencies of the time series. Moreover, in terms of \textbf{RQ 3.2} - \textit{What is the improvement achieved by employing \ac{STACK} approach coupled with signal decomposition approaches over non-decomposed models when forecasting wind speed time series?}, the divide-and-conquer scheme enhanced the accuracy of the base-learners, combining their strengths to form a more substantial model, giving more robustness and stability, even though some weak models presented high errors and low accuracy. Finally, regarding \textbf{RQ 6} - \textit{Can the employment of \ac{STACK} method improve the forecasting performance of the multi-stage signal decomposition strategy?}, the dual decomposition performance was enhanced by combining \ac{STACK} as a result of the divide-and-conquer scheme, same as resulted by the answers of the \textbf{RQ 3.2}. To answer \textbf{RQ 1.3}, \textbf{RQ 3.2}, \textbf{RQ 5}, and \textbf{RQ 6}, the specific objectives \textbf{\ref{obj_a}}, \textbf{\ref{obj_b}}, and \textbf{\ref{obj_e}} were achieved.

% Limitation of the applications
Despite the satisfactory predictions from the proposed forecasting frameworks, the applications experiments have some limitations. Regarding the first application, the forecasting framework was performed in the five most affected States of Brazil and \ac{USA}, respectively, by \ac{COVID-19}. However, it could be performed in other States and Countries from different Continents. Further, the experimentation limited the exogenous features only to climatic variables. The forecasting framework did not consider demographic factors, urban mobility information, and political rules. Regarding the last two applications, the experiments used data from consecutive months of the same year. Alternatively, the applications could compare non-consecutive months from different seasons (summer, fall, winter, and spring) to give different panoramas to observe if the model could accurately forecast regardless of the season. Also, all signal decomposition methods were trained homogeneously, i.e., the same forecasting model was performed in all modes. The forecasting framework could consider a heterogeneous approach to test if the different combinations of the modes' predictions result in better forecasting. Last, the third application did not use any exogenous features in the forecasting learning process due to own complexity of the proposed forecasting framework. Despite all this, these limitations do not invalidate the experiments performed or their results.

% Main contributions of the thesis
In summary of all discussions mentioned above, this thesis presented several findings and contributions. First, the use of signal decomposition methods indeed improves forecasting accuracy, which the experiments proved by applying them in diverse fields, such as epidemiological and renewable energy. Further, using exogenous features impacts the forecasting learning process in the time series. Moreover, the stacking-ensemble learning approach is essential in enhancing forecast accuracy once its divide-and-conquer strategy takes advantage of the diverse characteristics of each model chosen in its layers, making the model even more robust. It is important to highlight the diversity of the models has more impact on the learning process than the number of models applied. In other words, the more heterogeneous the ensemble is constructed, the more the ensemble could enhance forecasting accuracy. Additionally, using the signal decomposition methods in more than one stage improves forecasting accuracy even more than a single decomposing strategy. For this, performing a decomposition method to extract a specific mode of the time series (such as trend, seasonality, or other) and then decomposing the remaining with another approach indeed eases the learning process performed by the forecasting models. Last, combining all ideas presented in this thesis, employing the \ac{STACK} in the multi-stage decomposition strategy creates the most powerful hybrid forecasting framework that efficiently deals with nonlinear, non-stationary, and high fluctuations time series. This hybrid framework outperformed all other frameworks proposed in this thesis. Hence, considering all findings in this thesis and the exhaustive experiments, the hypothesis that combining multi-stage decomposition strategy with the stacking-ensemble learning approach improves the accuracy of forecasting models applied in real-world time series is supported and should not be rejected.

% Future works and research opportunities
Last, regarding analyzing the experiments proposed in this thesis, some research opportunities can be identified as follows.

\begin{enumerate}[label=(\roman*)]
    \item Using different forecasting models to compose the \ac{STACK} framework in both layers to compare the results with the previous experiments;
    \item Employing signal decomposition methods that have different decomposition strategies than the ones used in this thesis, such as Dynamic Mode Decomposition, \ac{EWT}, and Hilbert–Huang transform, among others;
    \item Adopting a heterogeneous combination in the reconstruction stage of the decomposition phase with different models, i.e., forecasting each mode with a different forecasting model and testing different combinations;
    \item Performing optimization algorithms to define the hyperparameters of the forecasting models, the hyperparameters of the signal decomposition methods, or the number of modes the time series would be decomposed;
    \item Using the TimeStacking, an improvement in the \ac{STACK} approach in which the method creates lag features from the base models predictions, proposed by \citeonline{ribeiro2022TimeStacking}. The idea is to combine it with the multi-stage decomposition strategy.
\end{enumerate}