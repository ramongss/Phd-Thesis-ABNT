\section{Motivation} \label{sec:motivation}

Given the gaps found in the literature (detailed in Chapter \ref{chap:relatedworks}) concerning the stacking-ensemble learning approach and signal decomposition methods, the combination of both strategies is even less explored. The contributions of this thesis, based on the evaluation of the signal decomposition methods coupled with the stacking-ensemble learning approach, justified its development.

The applications developed in this thesis presented how the proposed forecasting framework (the hypothesis of this thesis) could outperform simple forecasting strategies in different study fields. From the epidemiological area, which has the appeal of public health relevance, to forecasting renewable energy sources, which brings attention to corporate needs, the proposed framework showed exciting predictions and satisfactory results.

Hence, for these reasons, the combination of multi-stage decomposition strategy and stacking-ensemble learning approach could provide accurate forecasts that supply the needs of strategic planning in both the public and private sectors. Yet, since the contributions present a novel forecasting framework, this thesis also denotes academic relevance to the scientific community.