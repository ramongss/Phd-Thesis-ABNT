\section{Objectives} \label{sec:objectives}
To check and answer the hypothesis as mentioned earlier in this study and to fill the gaps in the literature regarding the applications of signal decomposition methods and \ac{STACK} approach for time series forecasting (detailed in Chapter \ref{chap:relatedworks}), this thesis proposes to investigate the efficiency of the approaches mentioned above by combining and employing them in real-world applications, such as forecasting (i) \ac{COVID-19} cases (epidemiological context), (ii) wind energy generation, and (iii) wind speed (renewable energy context). Regarding these contexts, each following \ac{RQ} is proposed:


\begin{enumerate}[start=1, label={\textbf{\ac{RQ} \arabic*}}, wide=0pt, leftmargin=3em]
    \item \label{rq1} Can signal decomposition approaches enhance the performance of forecasting epidemiological and renewable energy time series?

    \item \label{rq2} How does the use of exogenous features impact performance improvement when forecasting epidemiological time series?

    \item \label{rq3} What is the improvement achieved by employing \ac{STACK} approach coupled with signal decomposition approaches over non-decomposed models?

    \item \label{rq4} Can preprocessing methods applied to the time series improve the forecasting performance of the decomposition-ensemble learning strategy?

    \item \label{rq5} Can the multi-stage signal decomposition strategy outperform the single-stage signal decomposition approach?

    \item \label{rq6} Can the employment of \ac{STACK} method improve the forecasting performance of the multi-stage signal decomposition strategy?
\end{enumerate}

Summarizing the \ac{RQ}s mentioned, the main objective of this thesis is to evaluate the effectiveness of employing signal decomposition methods coupled with a stacking-ensemble learning approach for forecasting time series in real-world applications. Moreover, to achieve the main objective of the thesis, specific objectives are proposed as follows:

\begin{enumerate}[label={\alph*)}]
    \item \label{obj_a} Identifying research gaps regarding signal decomposition methods and \ac{STACK} approach for forecasting epidemiological and renewable energy time series;

    \item \label{obj_b} Comparing the forecasting performance of the signal decomposition methods coupled with \ac{STACK} approach over non-decomposed and non-ensemble models based on \ac{MAE}, \ac{MAPE}, \ac{RMSE}, \ac{RRMSE}, \ac{sMAPE}, and \ac{SSE} performance criteria. Further, \ac{DM} hypothesis test to compare the forecasting errors of the models;

    \item \label{obj_c} Investigating the importance of the exogenous features in the forecasting learning process;

    \item \label{obj_d} Measuring the improvement achieved by employing preprocessing methods in the time series in the forecasting learning process;

    \item \label{obj_e} Comparing the forecasting performance of the multi-stage signal decomposition strategy coupled with \ac{STACK} approach with the forecasting strategies proposed in the specific objective (b).  
\end{enumerate}

