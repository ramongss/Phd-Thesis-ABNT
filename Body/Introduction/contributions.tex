\section{Contributions} \label{sec:contributions}

In essence, this thesis has four significant contributions, which were presented by three applications. The first is evaluating the performance of employing signal decomposition methods in the forecasting learning process. The second contribution explores combining the previous techniques with the stacking-ensemble learning approach to enhance forecasting performance. Next, the third one evolves the first contribution by performing a multi-stage signal decomposition strategy and comparing them with single decomposed models. The last contribution lies in combining the \ac{STACK} approach to improve the forecasting performance of the multi-stage signal decomposition strategy. The summary of the contributions of this thesis is presented as follows.

The first contribution discussed the employment of signal decomposition methods. This contribution addresses \ref{rq1} and \ref{rq2}, as well the specific objectives \textbf{\ref{obj_a}} and \textbf{\ref{obj_c}}. The first application of this thesis regards this contribution. The results of this proposal were published in Chaos, Solitons \& Fractals Journal \cite{dasilva2020Forecasting}. In that study, the forecasting framework applied \ac{VMD} to decompose the \ac{COVID-19} cumulative cases time series in the Brazilian and American context. The five most affected States of each country (Brazil and \ac{USA}) at that time were used as forecast scenarios. To perform the forecast, \ac{BRNN}, \ac{CUBIST}, \ac{QRF}, and \ac{SVR} were adopted. The \ac{VMD}--based models were compared to non-decomposed models to evaluate the performance improvement of the proposed strategy. For that, \ac{sMAPE} and \ac{RRMSE} performance criteria. Moreover, the application evaluated exogenous features (climatic information) to help understand the forecasting learning process of the machine learning models. As the importance of each feature was retrieved, it was possible to understand the learning path of the decision made by the proposed model.

Second, the next contribution lies in advance of the previous framework by employing a \ac{STACK} approach in the forecasting learning process. This contribution aims to answer \ref{rq1}, \ref{rq3}, and \ref{rq4}, as well the specific objectives \textbf{\ref{obj_a}}, \textbf{\ref{obj_b}}, and \textbf{\ref{obj_d}}. For this, a second application was conducted, and its results were published in Energy Journal \cite{dasilva2021Novel}. In that case, the wind energy generation of a turbine in a wind farm in the Northeast Region of Brazil was considered the forecasting target. The framework is composed by \ac{CEEMD} and \ac{STACK} approach, as well \ac{BC}, \ac{CORR}, and \ac{PCA} to preprocess the data (addressing \ref{rq4}) and \ac{CUBIST}, \ac{KNN}, \ac{PLS}, \ac{RIDGE}, and \ac{SVR} as forecasting models. Further, the three preprocess techniques mentioned before were evaluated by comparing them with non-decomposed, ensemble models, and single models with performance criteria, such as \ac{MAE}, \ac{MAPE}, and \ac{RMSE}, as well \ac{DM} hypothesis test.

The last two contributions rely on the third application, which discusses the multi-stage decomposition strategy while extending the discussion by employing the \ac{STACK} approach in that framework. This application was published in the International Journal of Electrical Power \& Energy Systems \cite{dasilva2022Multistep}, which is an extension of the frameworks presented in the Energy Conversion and Management Journal \cite{moreno2020Multistep} and the 26th International Congress of Mechanical Engineering \cite{dasilva2021Multistep}. The third contribution addresses \ref{rq1} and \ref{rq5} (for multi-stage decomposition strategy), and the fourth contribution answers \ref{rq3} and \ref{rq6} (multi-stage decomposition plus \ac{STACK}). Both contributions also answer the specifics objectives \textbf{\ref{obj_a}}, \textbf{\ref{obj_b}}, and \textbf{\ref{obj_e}}. In summary, this application proposed a hybrid forecasting framework that employed \ac{VMD} to extract the trend component of the time series, and the \ac{SSA} method decomposed the remaining signal into four other components. Those five modes were predicted by \ac{CUBIST}, \ac{KNN}, \ac{PLS}, \ac{RIDGE}, and \ac{SVR}. The experimentation compared the proposed framework with several methods, such as dual decomposed, single decomposed, and non-decomposed models. Through these experiments, the proposed model could be effectively compared using \ac{IP}, \ac{MAE}, \ac{MAPE}, \ac{RMSE}, \ac{RRMSE}, and \ac{SSE} performance criteria, and \ac{DM} hypothesis test to evaluate the errors.