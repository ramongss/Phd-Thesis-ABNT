% resumo em portugu\^es
\setlength{\absparsep}{18pt} % ajusta o espa\c{c}amento dos par\'agrafos do resumo
\begin{resumo}[Resumo]
\begin{otherlanguage*}{brazil}

A previs\~ao de s\'eries temporais \'e uma abordagem essencial para empresas e pesquisadores tomarem decis\~oes embasadas, prevendo tend\^encias e padr\~oes futuros em dados de s\'eries temporais. 
%
No entanto, a previs\~ao precisa de s\'eries temporais pode ser desafiadora devido \`a complexidade dos dados, ma vez que pode apresentar flutua\c{c}\~oes elevadas e comportamento n\~ao-linear e n\~ao-estacion\'ario. 
%
Os modelos de aprendizado de m\'aquina s\~ao propostos para a previs\~ao de s\'eries temporais para superar essa limita\c{c}\~ao, pois podem capturar as rela\c{c}\~oes complexas e n\~ao-lineares nos dados. Em particular, os métodos de aprendizado em conjunto por empilhamento (\textit{stacking-ensemble learning}) e decomposi\c{c}\~ao de sinais s\~ao abordagens promissoras para realizar previs\~oes precisas.
%
Esta tese tem como objetivo avaliar a efic\'acia da utiliza\c{c}\~ao de m\'etodos de decomposi\c{c}\~ao de sinais combinados com uma abordagem de aprendizado em conjunto por empilhamento para a previs\~ao de s\'eries temporais em aplica\c{c}\~oes do mundo real. 
%
Para isso, tr\^es aplica\c{c}\~oes s\~ao apresentadas nesta tese. O primeiro caso utiliza a decomposi\c{c}\~ao de sinal para prever os casos acumulados de COVID-19 em cinco estados do Brasil e cinco estados dos Estados Unidos. O segundo caso utiliza uma abordagem de aprendizado em conjunto por empilhamento para prever a gera\c{c}\~ao de energia e\'olica de uma turbina em um parque e\'olico no Nordeste do Brasil. O terceiro caso estende a discuss\~ao utilizando uma estratégia de decomposi\c{c}\~ao de sinal em m\'ultiplos est\'agios e uma abordagem de aprendizado em conjunto por empilhamento para prever a velocidade do vento.
%
O desempenho das previs\~oes dos experimentos foi avaliado utilizando diferentes crit\'erios de desempenho, como erro absoluto m\'edio, erro percentual absoluto m\'edio, erro quadr\'atico m\'edio, erro quadr\'atico m\'edio relativo e soma de erros quadr\'aticos. O teste de hip\'otese de Diebold-Mariano foi realizado para determinar a signific\^ancia da diferen\c{c}a de erro nas previs\~oes analisadas. 
%
Analisando os resultados das experimenta\c{c}\~oes, foi poss\'ivel identificar que a estrat\'egia proposta de decomposi\c{c}\~ao em m\'ultiplos est\'agios combinada com a abordagem de aprendizado em conjunto por empilhamento poderia atingir erros de previs\~ao menores, superando modelos \'unicos, decompostos, n\~ao-decompostos e de aprendizado em conjunto.
%
A estrutura de previs\~ao proposta obteve erros inferiores a 3\% em alguns cen\'arios. Em comparação com as outras abordagens, a proposta apresentou uma melhoria m\'edia de desempenho variando entre 4,39\% e 63,67\%.
%
Portanto, considerando todas as descobertas nesta tese e os experimentos exaustivos, a hip\'otese de que a abordagem proposta melhora a precis\~ao dos modelos de previs\~ao de s\'eries temporais \'e suportada e n\~ao deve ser rejeitada.
    
\vspace{3mm}
\noindent \textbf{Palavras-chave:} Previs\~ao de S\'eries Temporais, Aprendizado de M\'aquina, Decomposi\c{c}\~ao de Sinal, Aprendizado de Comit\^es Empilhados. 
 
\end{otherlanguage*}
\end{resumo}